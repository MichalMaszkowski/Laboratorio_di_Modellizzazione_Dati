\subsection{Physical Model}\label{subsec:PhysicalModel}


In our model, we only consider the motion on the galactic plane ($b\approx0$). We assume that each star moves with a circular orbit around the Galactic Center GC with the same speed, $V_{\text{rot}}$. Fixing the frame of reference as in figure \ref{fig:FrameOfReference}, a generic star s has a velocity $\bm{v}_{\text{s}}$ given by:

\begin{equation}\label{eq:VComponents}
    \begin{aligned}
        \bm{v}_{\text{s}} &= V_{\text{rot}}\begin{pmatrix} \sin(\varphi) \\ -\cos(\varphi) \end{pmatrix}
    \end{aligned}
\end{equation}
\noindent
where $\varphi$ is the angle from the x-axis of the star with respect to the GC. In the following, primate vectors are in the frame of reference of the Sun, whereas unprimed ones are in the frame of reference of the GC. Angles are supposed to be expressed as radians.

In this frame of reference, the Sun has $\varphi = \pi$, and we assume it to be at a fixed distance $R=\qty{8300}{pc}$ with respect to the GC. In this model, the drift velocity of the LSR (Local Standard of Rest) is also given by equation \ref{eq:VComponents}. In addition, the Sun has its own peculiar motion with respect to the LSR, with components, $U_{\odot}$ along the $x$-axis, and $V_{\odot}$ along the $y$ one. Therefore, the total velocity $v_{\odot}$ of the Sun is:

\begin{equation}\label{eq:VSun}
    \bm{v}_{\odot} = \begin{pmatrix} 0 \\ V_{rot} \end{pmatrix} + \begin{pmatrix} U_{\odot} \\ V_{\odot} \end{pmatrix}
\end{equation}

\noindent
%The frame of reference of the Sun is moving with velocity $\bm{v}_{\odot}$ given by eq.\ref{eq:SunVelocity}, and its axis are rotating with an angular velocity $\bm{w}_{\odot} = -w_{\odot} \hat{e}_z$. 
The velocity $\bm{v'}_s$ of a star s, in the frame of reference of the Sun, at distance $d$ from the Sun is given by the equation:
\begin{equation}\label{eq:ReferenceFrame}
    \bm{v}_s' = \bm{v}_s - \bm{v}_{\odot} 
    %- \bm{w}_{\odot} \times \hat{e}_r' d = \bm{v}_s - \bm{v}_{\odot} + w_{\odot} \hat{e}_l' d
\end{equation}

\noindent 
We can now project the velocity $\bm{v'}_s$ onto radial direction with respect to the Sun $\hat{e}_r'$,
% , given by

% \begin{equation}\label{eq:radialverson}
%     \hat{e}_r' = \begin{pmatrix} \cos(l) \\ \sin(l) \end{pmatrix}
% \end{equation}
% \noindent
obtaining the radial component $v_{\text{rad}'}$ of the velocity of a star with longitude $l$ in the Sun's frame of reference:

\begin{equation}\label{eq:VSunRad}
    \begin{aligned}
        v_{\text{rad}'}  &= \bm{v}_s' \cdot \hat{e}_r' = \\ 
        &=V_{\text{rot}} \biggl[ \sin\varphi \cos l - (1 + \cos\varphi)\sin l \biggr] - U_{\odot} \cos l - V_{\odot} \sin l
    \end{aligned}
\end{equation}
\noindent
%Note that the rotation of the axis of the frame of reference of the Sun contributes only to the longitudinal component of the relative velocities of stars since $\hat{e}_r' \cdot \hat{e}_l' = 0$.

\input{CCTikzPicture.tex}

Eq.\ref{eq:VSunRad} must be adapted to the actual data provided by GAIA, which means expressing $sin\varphi$ and $cos\varphi$ in terms of $l$ and the parallax $p$, expressed in milliarcoseconds (\unit{mas}). First of all, the distance in parsec can be computed as:

\begin{equation}\label{eq:DistanceParallax}
    d[\unit{pc}] = \frac{1000}{p[\unit{mas}]}
\end{equation}

\noindent
Then, by geometric considerations we can evaluate $D, \sin\varphi$ and $\cos\varphi$ as a function of $R$, $d$ and $l$ (see figure \ref{fig:FrameOfReference}), obtaining:
\begin{equation}\label{eq:D}
    D=\sqrt{d^2 + R^2 - 2dR \cos l}
\end{equation}

\begin{equation}\label{eq:SinPhi}
    \sin\varphi = \pm \sqrt{1 - \cos^2 \varphi} = \frac{d\sin l}{D}
\end{equation}

\begin{equation}\label{eq:CosPhi}
    \cos\varphi = \frac{d \cos l - R}{D}
\end{equation}
\noindent
By substituting eq.\ref{eq:DistanceParallax}-\ref{eq:CosPhi} into eq.\ref{eq:VSunRad}, we get an expression for the prediction of the model for the radial component of the velocity of star $i$ $\hat{v}_{rad,i}^{(1)}(l_i, p_i)$ as a function of the measurements of its longitude and parallax $l_i, p_i$:

\begin{equation}\label{eq:VradModel1}
    \begin{aligned}
        \hat{v}_{rad,i}^{(1)}(l_i, p_i) =&  V_{\text{rot}}\sin l_i \biggl( \frac{R}{\sqrt{(\frac{1000}{p_i})^2 + R^2 - 2(\frac{1000}{p_i})R \cos l_i}}-1 \biggr) -\\
        &- U_{\odot}\cos l_i - V_{\odot}\sin l_i
    \end{aligned}
\end{equation}

In a second, more sophisticated model, we also consider the random motion of all stars, adding to the right-hand-side of equation~\ref{eq:VradModel1} a random variable $v_{\text{rand,i}} \sim N(0, \sigma)$, obtaining:
%distributed with null expected value and variance $\sigma^2$, in order to consider also the peculiar motion of each star, obtaining:

\begin{equation}\label{eq:VradModel2}
    \begin{aligned}
        \hat{v}_{rad,i}^{(2)}(l_i, p_i) =&  V_{\text{rot}}\sin l_i \biggl( \frac{R}{\sqrt{(\frac{1000}{p_i})^2 + R^2 - 2(\frac{1000}{p_i})R \cos l_i}}-1 \biggr) -\\
        &- U_{\odot}\cos l_i - V_{\odot}\sin l_i + v_{\text{rand,i}}
    \end{aligned}
\end{equation}