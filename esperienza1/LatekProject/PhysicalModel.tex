\subsection{Physical Model}\label{subsec:PhysicalModel}

In our model, we restrict attention to stellar motion confined to the Galactic plane ($b \approx 0$). We assume that each star moves in a circular orbit around the Galactic Center (GC) with a uniform speed $V_{\text{rot}}$. Fixing the coordinate system as shown in Figure~\ref{fig:FrameOfReference}, the velocity $\bm{v}_{\text{s}}$ of a generic star $s$ is given by:

\begin{equation}\label{eq:VComponents}
    \bm{v}_{\text{s}} = V_{\text{rot}} \begin{pmatrix} \sin\varphi \\ -\cos\varphi \end{pmatrix},
\end{equation}

\noindent
where $\varphi$ is the angle between the star’s position vector and the $x$-axis, measured from the GC. In the following, primed quantities refer to the Sun's frame of reference, while unprimed quantities are defined in the GC frame. All angles are expressed in radians.

\input{CCTikzPicture.tex}

In this reference frame, the Sun is located at $\varphi = \pi$, at a fixed distance of $R = \qty{8300}{pc}$ from the GC~\cite{GalacticKinematics}. The Local Standard of Rest (LSR) shares the same rotational velocity as described by Eq.~\ref{eq:VComponents}. Additionally, the Sun has a peculiar motion with respect to the LSR, with components $U_{\odot}$ along the $x$-axis and $V_{\odot}$ along the $y$-axis. The total velocity of the Sun is therefore:

\begin{equation}\label{eq:VSun}
    \bm{v}_{\odot} = \begin{pmatrix} 0 \\ V_{\text{rot}} \end{pmatrix} + \begin{pmatrix} U_{\odot} \\ V_{\odot} \end{pmatrix}.
\end{equation}

The velocity of a star in the Sun's frame is then given by:

\begin{equation}\label{eq:ReferenceFrame}
    \bm{v}_s' = \bm{v}_s - \bm{v}_{\odot}.
\end{equation}

We now project $\bm{v}_s'$ onto the radial direction relative to the Sun, denoted $\hat{e}_r'$, obtaining the observed radial velocity $v_{\text{rad}}'$ of a star with Galactic longitude $l$:

\begin{equation}\label{eq:VSunRad}
    \begin{aligned}
        v_{\text{rad}}' &= \bm{v}_s' \cdot \hat{e}_r' \\
        &= V_{\text{rot}} \left[ \sin\varphi \cos l - (1 + \cos\varphi)\sin l \right] - U_{\odot} \cos l - V_{\odot} \sin l.
    \end{aligned}
\end{equation}

To relate this model to Gaia observations, we must express $\sin\varphi$ and $\cos\varphi$ in terms of $l$ and the parallax $p$ (in milliarcseconds). The distance $d$ (in parsecs) from the Sun to a star is calculated as:

\begin{equation}\label{eq:DistanceParallax}
    d~[\unit{pc}] = \frac{1000}{p~[\unit{mas}]}.
\end{equation}

Using simple geometric relations (see Figure~\ref{fig:FrameOfReference}), we find the distance $D$ from the star to the GC, and express $\sin\varphi$ and $\cos\varphi$ as:

\begin{equation}\label{eq:Geometry}
    \begin{aligned}
        D &= \sqrt{d^2 + R^2 - 2dR \cos l}, \\
        \sin\varphi &= \frac{d \sin l}{D}, \\
        \cos\varphi &= \frac{d \cos l - R}{D}.
    \end{aligned}
\end{equation}

Substituting Eqs.~\ref{eq:DistanceParallax}--\ref{eq:Geometry} into Eq.~\ref{eq:VSunRad}, we obtain the prediction for the radial velocity of star $i$ under the first model, $\hat{v}_{\text{rad},i}^{(1)}$, as a function of its longitude $l_i$ and parallax $p_i$:

\begin{equation}\label{eq:VradModel1}
    \begin{aligned}
        \hat{v}_{\text{rad},i}^{(1)}(l_i, p_i) = &~V_{\text{rot}} \sin l_i \left( \frac{R}{\sqrt{\left(\frac{1000}{p_i}\right)^2 + R^2 - 2 \left(\frac{1000}{p_i}\right) R \cos l_i}} - 1 \right) \\
        &- U_{\odot} \cos l_i - V_{\odot} \sin l_i.
    \end{aligned}
\end{equation}

In a more refined second model, we account for the random peculiar motion of stars. We model each star's radial velocity component due to random motion as a Gaussian variable: $v_{\text{rand},i} \sim \mathcal{N}(0, \sigma)$, where $\sigma$ is the velocity dispersion. The corresponding prediction becomes:

\begin{equation}\label{eq:VradModel2}
    \begin{aligned}
        \hat{v}_{\text{rad},i}^{(2)}(l_i, p_i) = &~V_{\text{rot}} \sin l_i \left( \frac{R}{\sqrt{\left(\frac{1000}{p_i}\right)^2 + R^2 - 2 \left(\frac{1000}{p_i}\right) R \cos l_i}} - 1 \right) \\
        &- U_{\odot} \cos l_i - V_{\odot} \sin l_i + v_{\text{rand},i}.
    \end{aligned}
\end{equation}
