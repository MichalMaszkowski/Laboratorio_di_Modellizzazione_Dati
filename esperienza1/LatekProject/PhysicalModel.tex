\subsection{Physical Model}

In our model, the Sun's frame moves around the Galactic Center GC with a drift velocity (that of the Local Standard of Rest LSR) plus a random vector. The drift velocity of the LSR in this model is considered in module equal to the drift velocities of each star in our galaxy and analyzed by GAIA DR2.
In the following, primate vectors are in the frame of reference of the Sun, whereas unprimed ones are in the frame of reference of the center of the galaxy. Angles are supposed to be expressed as radians.

Calling $\bm{v}_{\odot}$ the total velocity of the Sun relative to the center of the galaxy, we have the following relation:

\begin{equation}\label{eq:SunVelocity}
    \bm{v}_{\odot} = \bm{v}_{\text{LSR}} + \bm{v}_{\text{rand}}
\end{equation}

\noindent
We can fix the frames of reference in the center of the Galaxy and on the Sun as in fig.\ref{fig:FrameOfReference}. In the picture, all the velocities are represented in the frame of reference fixed at the center of the galaxy. In our model, in this frame, all the stars (and the LSR frame) move around the center with velocity $V_{\text{rot}}$, therefore, the velocity for a star s at angle $\varphi$ from the $x$-axis is:

\begin{equation}\label{eq:VComponents}
    \begin{aligned}
        \bm{v}_s &= V_{rot} (-\hat{e}_{\varphi}) \\
        \hat{e}_{\varphi} &= \begin{pmatrix} -\sin(\varphi) \\ \cos(\varphi) \end{pmatrix}
    \end{aligned}
\end{equation}

\noindent
In particular, the Sun in this frame of reference has $\varphi = \pi$ and we add for it also $\bm{v}_{\text{rand}}$, as in equation \ref{eq:SunVelocity}, which is composed by two component, $U_{\odot}$ along the $x$-axis, and $V_{\odot}$ along the $y$ one. In this way its velocity is given by the equation:

\begin{equation}\label{eq:VSun}
    \bm{v}_{\odot} = \begin{pmatrix} 0 \\ V_{rot} \end{pmatrix} + \begin{pmatrix} U_{\odot} \\ V_{\odot} \end{pmatrix}
\end{equation}
 
\noindent
The frame of reference of the sun is moving with velocity $\bm{v}_{\odot}$ given by eq.\ref{eq:SunVelocity}, and its axis are rotating with an angular velocity $\bm{w}_{\odot} = -w_{\odot} \hat{e}_z$. Therefore, the velocity $\bm{v'}_s$ of a star s at distance $d$ from the Sun is given by the equation
\begin{equation}\label{eq:ReferenceFrame}
    \bm{v}_s' = \bm{v}_s - \bm{v}_{\odot} - \bm{w}_{\odot} \times \hat{e}_r' d = \bm{v}_s - \bm{v}_{\odot} + w_{\odot} \hat{e}_l' d
\end{equation}

\noindent 
The radial component $v_{\text{rad}'}$ of the velocity of a star with longitude $l$ in the sun frame of reference is finally given by:

\begin{equation}\label{eq:VSunRad}
    \begin{aligned}
        \hat{e}_r' &= \begin{pmatrix} \cos(l) \\ \sin(l) \end{pmatrix} \\
        v_{\text{rad}'}  &= \bm{v}_s' \cdot \hat{e}_r' = \\ 
        &=V_{\text{rot}} \biggl[ \sin\varphi \cos l - (1 + \cos\varphi)\sin l \biggr] - U_{\odot} \cos l - V_{\odot} \sin l
    \end{aligned}
\end{equation}
\noindent
%Note that the rotation of the axis of the frame of reference of the Sun contributes only to the longitudinal component of the relative velocities of stars since $\hat{e}_r' \cdot \hat{e}_l' = 0$.


\input{CCTikzPicture.tex}

Eq.\ref{eq:VSunRad} must be adapted to the actual data provided by GAIA, which means expressing $sin\varphi$ and $cos\varphi$ in terms of $l$ and the parallax $p$, expressed in milliarcoseconds (\unit{mas}). First of all, the distance in parsec can be computed as:

\begin{equation}\label{eq:DistanceParallax}
    d[\unit{pc}] = \frac{1000}{p[\unit{mas}]}
\end{equation}

\noindent
Then, by applying the cosine theorem two times for $R, d, D, l, \varphi$ (fig.\ref{fig:FrameOfReference}), we get:
\begin{equation}\label{eq:D}
    D=\sqrt{d^2 + R^2 - 2dR \cos l}
\end{equation}

\begin{equation}\label{eq:CosPhi}
    \cos\varphi = \frac{d \cos l - R}{D}
\end{equation}

\noindent
and, therefore,

\begin{equation}\label{eq:SinPhi}
    \sin\varphi = \pm \sqrt{1 - \cos^2 \varphi} = \frac{d\sin l}{D}
\end{equation}
\noindent
By substituting eq.\ref{eq:DistanceParallax}-\ref{eq:SinPhi} into eq.\ref{eq:VSunRad}, we get an expression for the prediction of the model for the radial component of the velocity of star $i$ $v_{rad,i}^{mod}(l_i, p_i)$ as a function of the measurements of its longitude and parallax $l_i, p_i$:

\begin{equation}\label{eq:VradModel}
    $v_{rad,i}^{mod}(l_i, p_i)$ =  V_{\text{rot}}\sin l_i (\frac{R}{\sqrt{(\frac{1000}{p_i})^2 + R^2 - 2(\frac{1000}{p_i})R \cos l_i}}-1) - U_{\odot}\cos l_i - V_{\odot}\sin l_i
\end{equation}
