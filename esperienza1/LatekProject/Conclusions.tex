\section{Conclusions}

The median values and 95\% confidence intervals of the posterior distributions for both models are reported in Table~\ref{tab:parameter_estimates}. The estimates for $U_\odot$ and $V_\odot$ are consistent across the two models, while the estimates for $V_{\text{rot}}$ differ significantly — their respective 95\% confidence intervals do not overlap.

The inferred median values of $U_\odot$ from both models are in good agreement with results from the literature~\cite{LocalKinematics}, while the estimates of $V_\odot$ are notably higher and incompatible with those reported in the same source. Regarding $V_{\text{rot}}$, our values are lower than the literature estimate~\cite{GalacticKinematics}, but both models yield confidence intervals that partially overlap with the reference value, indicating general consistency.

The posterior distributions for all parameters are presented in Appendix~\ref{Appendix:PosteriorDistributions}. Additionally, the middle and lower panels of Figure~\ref{fig:DataModelPresentation} show the model predictions based on the median posterior values, using the longitude and parallax of the observed stars. For the second model, the effect of stellar random motion was simulated by adding a Gaussian noise term $v_{\text{rand}, i} \sim \mathcal{N}(0, \sigma)$, using the inferred median  value of $\sigma$.

Visually, it is evident that the second model provides a better fit to the data, highlighting the importance of incorporating the random motion of stars to accurately model their observed radial velocities.

\begin{table}[H]
    \centering
    \begin{tabular}{l c c | c}
        \hline
        Parameter & Model 1 & Model 2 & Literature \\
        \hline
        $V_{\text{rot}}$ [\unit{\kilo\meter\per\second}] & $211.45 \pm 0.04$ & $204 \pm 2$ & $225 \pm 20$~\cite{GalacticKinematics} \\
        $U_{\odot}$ [\unit{\kilo\meter\per\second}] & $11.638 \pm 0.005$ & $11.7 \pm 0.3$ & $11.1^{+0.7}_{-0.8}$~\cite{LocalKinematics} \\
        $V_{\odot}$ [\unit{\kilo\meter\per\second}] & $21.604 \pm 0.005$ & $21.7 \pm 0.3$ & $12.2 \pm 0.5$~\cite{LocalKinematics} \\
        $\sigma$ [\unit{\kilo\meter\per\second}] & --- & $30.6 \pm 0.2$ & --- \\
        \hline
    \end{tabular}
    \caption{%Estimated parameters and their 95\% confidence intervals for the two models and the corresponding values found in literature. The literature value of $V_{\text{rot}}$ has total uncertainty (1$\sigma$) given by the sum in quadrature of a statistical and a systematic contributions as reported in~\cite{GalacticKinematics}, we reported the double of this value in the table, for consistency with all the other data, presenting the limits of their 95\% confidence interval ($\pm 2\sigma$). All the data reported has, up to the first relevant decimal point, symmetrical 95\% confidence intervals, save exception for the literature value of $U_{\odot}$.
    Estimated values and their 95\% confidence intervals of the kinematic parameters of the two models, compared with values reported in the literature. The literature value of $V_{\text{rot}}$ is reported with a total uncertainty corresponding to $2\sigma$, to maintain consistency with the 95\% confidence intervals of all the other values. In the original article~\cite{GalacticKinematics}, its total uncertainty (1$\sigma$  = 10 \unit{\kilo\meter\per\second}) is computed as the quadrature sum of its statistical and systematic systematic components. All reported uncertainties are symmetric up to the first significant digit, except for the literature value of $U_{\odot}$, which retains its original asymmetric interval from~\cite{LocalKinematics}.}\label{tab:parameter_estimates}
\end{table}
