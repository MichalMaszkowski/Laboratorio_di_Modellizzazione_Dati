\subsection{Statistical Model}\label{subsec:StatisticalModel}

In this section, we describe the Bayesian inference framework employed to estimate the parameter distributions of the two models. The unnormalized posterior probability distributions were sampled using a Markov Chain Monte Carlo (MCMC) method implemented via the \texttt{emcee}~\cite{EMCEE} Python package. Specifically, we supplied \texttt{emcee} with the unnormalized posterior distribution, i.e., the numerator of Bayes' theorem:

\begin{equation}\label{Eq:NonNormPosterior}
    \Posterior^{*}(\theta \vert \mathcal{D}) = \Like(\mathcal{D} \vert \theta) \Prior(\theta) \text{ ,}
\end{equation}

\noindent where $\Like(\mathcal{D} \vert \theta)$ is the likelihood and $\Prior(\theta)$ the prior. In practice, their logarithms were used to enhance numerical stability. Constant factors independent of the model parameters were omitted, as they do not affect the sampling of the posterior distribution and are thus irrelevant for our purposes.

In both models, we assume all measurements and parameters to be statistically independent, given no a priori reason to consider correlations among them. This allows the likelihood and prior to be factorized into products over individual terms, and their logarithms expressed as corresponding sums. We model each measured quantity $q_{\text{measured}}$ as the sum of its true value $q_{\text{true}}$ and a random error $\epsilon$ drawn from a normal distribution centered at 0, with standard deviation $\sigma_{\text{q}}$, given by the reported statistical uncertainty provided by the GAIA database: $\epsilon \sim N(0, \sigma_\text{q})$. Thus, $q_{\text{measured}} = q_{\text{true}} + \epsilon$.

For the first model, the parameter vector is $\mathbf{\theta_1} = (V_{\text{rot}}, U_{\odot}, V_{\odot})$. Neglecting uncertainties on parallax measurements, the difference between the observed radial velocity $v_{\text{rad}, i}$ and its model prediction $\hat{v}^{(1)}_{\text{rad}, i}$ is treated as a Gaussian random variable centered at 0, with standard deviation given solely by the reported radial velocity uncertainty $\sigma_{\text{v},i}$. The log-likelihood for this model is therefore:

\begin{equation}\label{Eq:LogLikeMod1}
    \log \Like^{(1)}(\mathcal{D} \vert \theta_1) = -\frac{1}{2}\sum_{i=1}^{\text{N}_{\text{stars}}}\biggl[\log(2\pi\sigma_{\text{v},i}^2)+\frac{(v_{\text{rad},i} - \hat{v}_{\text{rad},i}^{(1)})^2}{\sigma_{\text{v},i}^2}\biggr] \text{ .}
\end{equation}

We assume the three parameters to be mutually independent. For $V_{\text{rot}}$, we adopt a flat prior over the interval $[0,\qty{500}{\kilo\meter\per\second}]$, which covers the typical rotational velocities in Sb galaxies~\cite{Schneider2015}, found in the range [144,~330]~\unit{\kilo\meter\per\second}\cite{Schneider2015}. For $U_\odot$ and $V_\odot$, we use Gaussian priors centered at 0, based on the assumption that the Sun's peculiar motion is akin to a stochastic thermal motion. Therefore, we model their log-prior to be proportional to:
\begin{equation*}
    \log\mathbb{P}^{(1)}(U_\odot) + \log\mathbb{P}^{(1)}(V_\odot) \sim - \frac{U_\odot^2 + V_{\odot}^2}{v_{\text{gal}}^2} \text{ ,}
\end{equation*}
\noindent
with $v_{\text{gal}} = \qty{200}{\kilo\meter\per\second}$, reflecting the typical velocity scale in spiral galaxies~\cite{Schneider2015}, as previously mentioned.

In the second model, the parameter vector is $\mathbf{\theta_2} = (V_{\text{rot}}, U_{\odot}, V_{\odot}, \sigma)$. In this case, uncertainties in both radial velocity and parallax are taken into account. 
The model prediction \(\hat{v}^{(2)}_{\text{rad}, i}\) (eq.\ref{eq:VradModel2}) is treated as a random variable drawn from a normal distribution whose variance includes two contributions: the intrinsic dispersion \(\sigma^2\), which is a parameter of the model, and the uncertainty due to parallax measurement errors.
Assuming the parallax error is small, its contribution to the model variance can be approximated by \(\left( \frac{\partial \hat{v}^{(2)}_{\text{rad}, i}}{\partial p_i} \right)^2 \sigma^2_{p_i}\), where the derivative is analytically derived from eq.\ref{eq:VradModel2}. 
The difference between the measured radial velocity \(v_{\text{rad}, i}\) and the model prediction \(\hat{v}^{(2)}_{\text{rad}, i}\) is then modeled as a Gaussian random variable centered at zero, with total variance equal to the sum of the variances of the radial velocity measurement, the parallax-induced component, and the intrinsic dispersion:

\begin{equation}\label{eq:ErrorPropagation}
    \sigma^2_{\text{tot}, i} = \sigma^2_{\text{v},i} +  {\left( \frac{\partial \hat{v}^{(2)}_{\text{rad}, i}}{\partial p_i} \right)}^2 \sigma^2_{p_i} + \sigma^2 \text{ .}
\end{equation}


The log-likelihood is therefore:
\begin{equation}\label{Eq:LogLikeMod2}
    \log \mathbb{\Like}^{(2)}(\mathcal{D} \vert \theta_2) = -\frac{1}{2}\sum_i\left\{\log[2\pi(\sigma_{i}^2+\sigma^2)]+\frac{(v_{\text{rad},i} - \hat{v}_{\text{rad},i}^{(2)})^2}{\sigma_{\text{tot}, i}^2}\right\} \text{ .}
\end{equation}
\noindent
The prior distributions for $V_{\text{rot}}$, $U_{\odot}$, and $V_{\odot}$ are retained from the first model. For the additional parameter $\sigma$, we assume it is uncorrelated with the others and assign it the non-informative prior appropriate for the standard deviation in Gaussian models, since it has a similar interpretation in the model we implemented: $\log \Prior (\sigma) = - \log(\sigma)$~\cite{mackay2003}.
