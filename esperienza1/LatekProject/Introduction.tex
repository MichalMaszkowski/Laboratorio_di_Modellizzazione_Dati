\section*{Abstract}

%In this excercise we aim to model the rotation of stars within the Milky Way around its center. 
In this report we present a model for the rotation of stars in the Milky Way around its center.
%We introduce an inertial frame of reference centered at the center of the galaxy, with xy-plane coinciding with the galactic plane and x-axis pointing away from the Sun. We will denote it as CG frame.
Firstly (in a simpler model), we assume that the stars, as well as the LSR (Local Standard of Rest), 
move in circular orbits around the origin of the GC (Galactic Center) frame 
with a constant velocity $V_{\text{rot}}$ tangential to the orbit,
and that the Sun is moving with respect to the LSR 
with a velocity $U_{\odot}$ in the x-direction and $V_{\odot}$ in the y-direction in the GC frame. 
Then, we present a more complex model, in which for each star, 
a random component $v_{\text{rand}}' \sim N(0, \sigma)$ is added to its velocity.

%We are going to assume that also LSR (Local Standard of Rest) is moving with a velocity $V_{rot}$ in the y-direction in the CG frame.
%We also introduce a frame of reference centered at the Sun, with the x-axis pointing towards the center of the galaxy, and the z-axis pointing towards the North Galactic Pole. We will denote it as Sun frame.
%The Sun frame is rotating with an angular velocity $w$ around the z-axis due to the rotation of the galaxy with respect to the CG frame.

In both cases we aim to model the velocity of stars in the radial direction 
in the Sun frame of reference.
% and find $V_{rot}$, $U_0$ and $V_0$ using Baysian Inference on the data provided by the GAIA mission.

In this report, we use data provided by GAIA DR2~\cite{GAIADR2}.
For each star on the galactic plane we consider its longitude, parallax, and radial velocity,
with respect to GAIA, which we equate with the Sun frame of reference.
We also consider the provided uncertainties of the measurements of parallax and radial velocity.

By means of Bayesian inference on the data we find the distributions 
of parameters of the two models, 
namely $\mathbf{\theta_1} = (V_{rot}, U_{\odot}, V_{\odot})$ 
and $\mathbf{\theta_2} = (V_{rot}, U_{\odot}, V_{\odot}, \sigma)$.
We used MCMC (Monte Carlo Markov Chain) 
to estimate the non-normalized posterior of the two models. 

In the first model, we estimate the parameters and their 95\% confidence interval to be: 
$V_{\text{rot}} = 211.45$, 95\% CI~=~[211.42,~211.50] \unit{\kilo\meter\per\second}; 
$u_{\odot} = 11.638$, 95\% CI~=~[11.633,~11.644]\unit{\kilo\meter\per\second}; 
$v_{\odot} = 21.604$, 95\% CI~=~[21.599,~21.609] \unit{\kilo\meter\per\second}. 
With the second one instead, we got: 
$V_{\text{rot}}= 204$, 95\% CI =~[202,~206] \unit{\kilo\meter\per\second}; 
$u_{\odot} = 11.7$, 95\% CI~=~[11.4,~12.0] \unit{\kilo\meter\per\second}; 
$v_{\odot} = 21.7$, 95\% CI~=~[21.4,~22.1] \unit{\kilo\meter\per\second}; 
$\sigma = 30.6$, 95\% CI~=~[30.5,~30.8] \unit{\kilo\meter\per\second}.