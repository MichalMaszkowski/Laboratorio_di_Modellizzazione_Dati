\section*{Abstract}
% In this report we present a model for the rotation of stars in the Milky Way around its center.
In this report, we present two Bayesian models for the rotational motion of stars in the Milky Way galaxy, 
using observational data from the Gaia Data Release 2 (DR2)~\cite{GAIADR2}. 
% Firstly, in a simpler model, we assume that the stars, as well as the Local Standard of Rest (LSR), 
% move in circular orbits around the origin of the Galactic Center (GC) frame with a constant velocity $V_{\text{rot}}$, 
% and that the Sun is moving with respect to the LSR with a velocity $U_{\odot}$ in the x-direction 
% and $V_{\odot}$ in the y-direction in the GC frame. 
We begin by introducing a simplified physical model in which both the stars and the Local Standard of Rest (LSR) 
follow circular orbits around the Galactic Center (GC) with a constant speed $V_{\text{rot}}$. 
The Sun is modeled as moving relative to the LSR with components $U_{\odot}$ (in the $x$-direction) 
and $V_{\odot}$ (in the $y$-direction), all defined within the GC frame. 
% Then, we present a more complex model, in which for each star, a random component $v_{\text{rand}}' \sim N(0, \sigma)$ is added to its velocity.
We then extend this model by introducing a random velocity component $v_{\text{rand}}' \sim \mathcal{N}(0, \sigma)$ for each star, to account for intrinsic stellar motion not captured by the pure circular rotation.

% %In both cases we aim to model the velocity of stars in the radial direction in the Sun frame of reference. 
% Applying Bayesian inference on the data provided by GAIA Data Release 2 (DR2)~\cite{GAIADR2}, we find the distributions of parameters of the two models, namely $\mathbf{\theta_1} = (V_{\text{rot}}, U_{\odot}, V_{\odot})$ and $\mathbf{\theta_2} = (V_{\text{rot}}, U_{\odot}, V_{\odot}, \sigma)$.
% For each star on the galactic plane we consider its measured longitude, parallax, and radial velocity, with respect to the Sun frame of reference.
% We also consider the provided uncertainties of the measurements of parallax and radial velocity.

% We utilized the Monte Carlo Markov Chain (MCMC) implementation from the emcee~\cite{EMCEE} package in Python to estimate the posterior distributions of both models.
Using Bayesian inference and applying Markov Chain Monte Carlo (MCMC) sampling via the \texttt{emcee}~\cite{EMCEE} package, 
we estimate the posterior distributions of the model parameters. 
The inference is performed using stellar longitude, parallax, and radial velocity measurements, 
along with their associated uncertainties, all expressed in the Sun-centered reference frame.

% In the first one, we estimate the parameters and their 95\% confidence interval to be: % put this into a table in Data analysis:
For the simpler model, we estimate the parameters and their 95\% confidence intervals as:  
$V_{\text{rot}} = 211.45 \pm 0.04$ \unit{\kilo\meter\per\second};  
$U_{\odot} = 11.638 \pm 0.005$ \unit{\kilo\meter\per\second};  
$V_{\odot} = 21.604 \pm 0.005$ \unit{\kilo\meter\per\second}.  

% With the second model instead, we got: 
In the extended model, which includes the velocity dispersion of all stars, we find: 
$V_{\text{rot}} = 204 \pm 2$ \unit{\kilo\meter\per\second};  
$U_{\odot} = 11.7 \pm 0.3$ \unit{\kilo\meter\per\second};  
$V_{\odot} = 21.7 \pm 0.3$ \unit{\kilo\meter\per\second};  
$\sigma = 30.6 \pm 0.2$ \unit{\kilo\meter\per\second}.  

