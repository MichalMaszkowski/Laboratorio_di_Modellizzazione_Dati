\section*{Abstract}
In this report we present a model for the rotation of stars in the Milky Way around its center.
Firstly, in a simpler model, we assume that the stars, as well as the Local Standard of Rest (LSR), 
move in circular orbits around the origin of the Galactic Center (GC) frame with a constant velocity $V_{\text{rot}}$, and that the Sun is moving with respect to the LSR with a velocity $U_{\odot}$ in the x-direction and $V_{\odot}$ in the y-direction in the GC frame. 
Then, we present a more complex model, in which for each star, a random component $v_{\text{rand}}' \sim N(0, \sigma)$ is added to its velocity.

%In both cases we aim to model the velocity of stars in the radial direction in the Sun frame of reference. 
Applying Bayesian inference on the data provided by GAIA Data Release 2 (DR2)~\cite{GAIADR2}, we find the distributions of parameters of the two models, namely $\mathbf{\theta_1} = (V_{\text{rot}}, U_{\odot}, V_{\odot})$ and $\mathbf{\theta_2} = (V_{\text{rot}}, U_{\odot}, V_{\odot}, \sigma)$.
For each star on the galactic plane we consider its measured longitude, parallax, and radial velocity, with respect to the Sun frame of reference.
We also consider the provided uncertainties of the measurements of parallax and radial velocity.

We utilized the Monte Carlo Markov Chain (MCMC) implementation from the emcee~\cite{EMCEE} package in Python to estimate the posterior distributions of both models.

In the first one, we estimate the parameters and their 95\% confidence interval to be: % put this into a table in Data analysis:
$V_{\text{rot}} = 211.45_{-0.04}^{+0.04}$ \unit{\kilo\meter\per\second};  
$U_{\odot} = 11.638_{-0.005}^{+0.005}$ \unit{\kilo\meter\per\second};  
$U_{\odot} = 21.604_{-0.005}^{+0.005}$ \unit{\kilo\meter\per\second}.  

With the second model instead, we got:  
$V_{\text{rot}} = 204_{-2}^{+2}$ \unit{\kilo\meter\per\second};  
$U_{\odot} = 11.7_{-0.3}^{+0.3}$ \unit{\kilo\meter\per\second};  
$U_{\odot} = 21.7_{-0.3}^{+0.3}$ \unit{\kilo\meter\per\second};  
$\sigma = 30.6_{-0.2}^{+0.2}$ \unit{\kilo\meter\per\second}.  

