\section{Introduction}

In this excercise we aim to model the rotation of stars within the Milky Way around its center. 
We introduce an inertial frame of reference centered at the center of the galaxy, 
with xy-plane coinciding with the galactic plane and x-axis pointing away from the Sun. We will denote it as CG frame.
We are going to assume that the stars in the CG frame move in circular orbits around its origin with a constant velocity $V_{rot}$.
We are going to assume that also LSR (Local Standard of Rest) is moving with a velocity $V_{rot}$ in the y-direction in the CG frame.
We also introduce a frame of reference centered at the Sun, with the x-axis pointing towards the center of the galaxy, 
and the z-axis pointing towards the North Galactic Pole. We will denote it as Sun frame.
The Sun frame is rotating with an angular velocity $w$ around the z-axis due to the rotation of the galaxy with respect to the CG frame.
We are going to assume that Sun frame is moving with a velocity $U_0$ in the x-direction and $V_0$ in the y-direction in the CG frame with respect to the LSR.

Our goal is to model the velocity of stars in the radial direction in the Sun frame of reference, 
and find $V_rot$, $U_0$ and $V_0$ using Baysian Inference on the data provided by the GAIA mission. 
The data contains the parallax and the radial motion as well as the longitude of stars with respect to the Sun frame of reference.